\chapter{Design Rationale}

\section{Virtual Reality}
    A virtual reality interface was chosen due to the immersive experience it provides. Users of the system are practicing for real-life situations, and in order to get the most out of the experience, the system must be as close to real life as possible. Virtual reality provides the closest experience to reality, without any keyboard controls to interface through. Moving through a virtual environment provides valuable experience for reacting to the real-life situations this program hopes to train people in.
\section{Google Cardboard}
    Google Cardboard technology was chosen due to the low cost of usage, providing a more cost-effective solution to virtual reality hardware than many of the major competitors, while still giving a quality experience. By offloading the software requirements to a smartphone and keeping the hardware down to a cardboard frame, Google Cardboard lets the product scale more effectively among users and increases accessibility.
\section{Biofeedback}

\section{Event-Based Architecture}