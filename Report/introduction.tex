\chapter{Introduction} % Jake [DONE]

\section{Motivation}

Parents, specifically those of at-risk youth, are not providing their children with responsive caregiving. 
These children are growing up in an environment in which their parents have low emotional resilience, a quality developed early on in one's childhood. 
When parents display this behavior in front of their children, these children grow up susceptible to stress and emotional distress, perpetuating a vicious cycle. 
Healthy parenting and family resilience in early childhood has been shown to be an important factor in effectively managing stress, promoting school readiness and achievement, and preventing adolescents from participating in high-risk behaviors.

At this time, there lacks a means to teach emotional resilience to reach a wide target audience. 
Traditional solutions such as psychotherapy or medication prescribed by a doctor may ameliorate the symptoms of stress, but fail to address the underlying issue. 
Even with the emergence of electronic wearable devices (i.e. stress detectors), this solution alone fails to help as they simply alert individuals if they are stressed, potentially worsening symptoms with constant reminders. 
Dr. Barbara Burns, professor and director of child studies at Santa Clara University (SCU), presents the most promising solution at this time; her Resilient Families Program (RFP) aims to promote family and community resilience with community-led, science-based parent education programs.
However, the program's outreach is severely limited to the number of individuals Dr. Burns can train at a given time. 

A former senior design project collaborated with Dr. Burns and came up with a VR Empathy Training Tool. 
The user can interact with a crying child and therefore learn how to handle stress (under certain circumstances). 
However, the project has the same level of difficulty for all individuals regardless of the difference of their emotional resilience. 
The project also lacks a sense of immersion and virtual "presence", which is a key component in creating empathy within VR. 
Furthermore, their design decision to measure stress using only heart rate is non-comprehensive.

Fortunately, resilience is not a fixed characteristic; it can be learned. 
As individuals, we can build an awareness of the situations in which we are least resilient and focus our efforts on developing personal resilience there. 

\section{Solution}

Our solution builds upon the framework laid out by RFP. We propose a virtual reality (VR) mobile application that allows parents to interact with a virtual child. 
The child will display signs of distress and will react based upon the user's decision. 
Based on prior research, we also believe allowing parents to take the perspective of the child will expedite their development toward empathy. 
Unlike prior solutions, ours does more than inform users they are stressed. 
The intensity of the experience will vary based on perceived stress levels.
This teaches users how to handle stress in a productive manner without overwhelming them from the start.
To address the lack of outreach of RFP, our solution will utilize an accessible and low cost VR implementation.
Our solution will also provide users with the ability to track their progress, based on stress levels exhibited during each session.
As a result, different individuals can get personalized training.
